\documentclass[a4paper,10pt]{article}

\usepackage[round]{natbib}

\title{ClonalOrigin in BEAST 2}
\author{Tim Vaughan}

\begin{document}

\maketitle{}

\section{Model}

The model implemented in this package is inspired by the approximation
to the coalescent with gene conversion \citep{Wiuf2000a} used by
\cite{Didelot2010} in their package ClonalOrigin. 

The model is used to find the joint posterior probability density of the
genealogy $G$ of the sample, the coalescent parameters $\theta$, the
substitution model parameters $\mu$ and the recombination rate $\rho$
conditional on the sequence alignment $A$.  This density can be
expanded in the following way
\begin{equation}
f(G,\theta,\mu,\rho|A) \propto P_{\mathrm{F}}(A|G,\mu)f_{\mathrm{CO}}(G|\rho,\delta,\theta)f_{\mathrm{prior}}(\theta,\rho,\mu)
\end{equation}
where $P_{\mathrm{F}}$ is Felsenstein's likelihood for the marginal
trees in $G$ under a substitution model with parameters $\mu$,
$f_{\mathrm{CO}}$ is the density for the gene conversion genealogy
under a modification of the ClonalOrigin model, and
$f_{\mathrm{prior}}$ is the prior density on the continuout
parameters.

We expand the genealogy density in the following way
\begin{equation}
f_{\mathrm{CO}}(G|\rho,\delta,\theta)=f(R|T,M,\theta)P(M|T,\rho,\delta)f_{\mathrm{C}}(T|\theta).
\end{equation}
The density $f_{\mathrm{C}}(T|\theta)$ is simply the density of the
clonal frame $T$ under the standard coalescent model, while the function
$f(R|T,M,\theta)$ is the density of the recombinant edges in the
graph conditional on the number of such edges (via $M$).  These edges
depart from and rejoin $T$ backward in time.  The departure points are
chosen uniformly over the tree, while their points of return are
chosen from conditional coalescent---giving the dependence on
$\theta$.

As it is the probability of the converted region set, $M$, which forms
our point of departure from the ClonalOrigin model, we describe it in
detail here.  We can write $M=\{(x_i,y_i)|i\in[1,q]\}$ with $q$ being
the number of recombinations affecting the sample. The elements of
each ordered pair $(x_i,y_i)$ define the bounds of the region affected
by event $i$. Contrary to \cite{Didelot2010}, we assume that these
regions are non-overlapping and can thus be ordered so that
$x_1<y_1<\ldots<x_q<y_q$. We apply the constraints $y_1\geq1$ and
$x_q\leq L$, although $x_1$ and $y_q$ are permitted to lie outside the
sampled sequence.

Allowing for both a constant recombination rate $\rho$ over the clonal
frame and a constant average conveted tract length is impossible under
the non-overlapping assumption.  Thus, we instead write
\begin{equation}
%P(M|T,\rho,\delta)= (1-\delta)^{-l}\left(1-\frac{\rho
%\lambda_T}{2}\right)^{L-l}\left(\frac{\rho\lambda_T}{2\delta}\right)^q
P(M|T,\rho,\delta) = p
\end{equation}
where $\lambda_T$ is the sum of the length of all edges of the clonal
frame, and $l\equiv\sum_{i=1}^q (y_i-x_i)$ is the total number of
converted loci. Note that we have made the additional assumption that
the 

\begin{table}
\begin{tabular}{|cl|}
  \hline
  Parameter & Definition \\
  \hline
  $A$ & Sequence alignment \\
  $G=\{T,R\}$ & Recombination graph \\
  $T$ & Clonal frame tree \\
  $R$ & Additional edges representing recombinations \\
  $M$ & Set of non-overlapping converted regions of alignment \\
  $\rho$ & Recombination rate (events per unit time) \\
  $\delta$ & Average converted tract length \\
  $\theta$ & Coalescent rate parameters (e.g. population size model) \\
  $\mu$ & Substitution model parameters \\
  \hline
\end{tabular}
\caption{Description of all parameters used in the model.}
\end{table}

%\bibliography{papers}
%\bibliographystyle{plainnat}

\begin{thebibliography}{2}
\providecommand{\natexlab}[1]{#1}
\providecommand{\url}[1]{\texttt{#1}}
\expandafter\ifx\csname urlstyle\endcsname\relax
  \providecommand{\doi}[1]{doi: #1}\else
  \providecommand{\doi}{doi: \begingroup \urlstyle{rm}\Url}\fi

\bibitem[Didelot et~al.(2010)Didelot, Lawson, Daarling, and
  Falush]{Didelot2010}
Xavier Didelot, Daniel Lawson, Aaron Daarling, and Daniel Falush.
\newblock Inference of homologous recombination in bacteria using whole-genome
  sequences.
\newblock \emph{Genetics}, 186:\penalty0 1435, 2010.

\bibitem[Wiuf and Hein(2000)]{Wiuf2000a}
C.~Wiuf and J.~Hein.
\newblock The coalescent with gene conversion.
\newblock \emph{Genetics}, 155\penalty0 (1):\penalty0 451--462, May 2000.

\end{thebibliography}


\end{document}